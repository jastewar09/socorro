\documentclass[letterpaper]{article}

\title{Socorro: High Level Functionality}
\author{Stephen Foiles}

\begin{document}

\section{Structural Optimization:}

Fixed volume optimization of the atomic positions. All atoms are
moved - there is currently no method for constraining specific
atoms. This is called by the statement "if
(optimize\_lattice(cfg)) call diary(cfg)" in socorro.f90.
optimize\_lattice returns a logical that indicates whether any
modifications to cfg have occurred.  The cfg that is returned
corresponds to the optimized positions.

The relevant options in argvf are as follows

\begin{list}{}{}
\item[relax\_method:] Specify the relaxation method to be used.

\begin{list}{}{}

\item[NONE] default Do not perform a structural optimization
\item[STEEPEST\_DESCENTS, SD]  'steepest descents' (I have seen
different definitions of 'steepest descents'.) What is implemented
here is the simple approach of changing the coordinates by  where
F is the current force and relax\_prefactor is a parameter that
you can set (see below).  This is usually NOT the best way to get
to a minimum.  It was included because it was the obvious first
thing to code and is a standard 'brute force' approach to
optimization.
\item[CONJUGATE\_GRADIENT, CG]  'conjugate
gradients' Implements a conjugate gradient search for the minimum
energy structure.  (See Press, et.al, 'Numerical Recipes' for a
description of this algorithm.)  This is typically the best way to
go when the initial positions are close to the minimum.
\item[QUENCHED\_MINIMIZATION, QM] 'quenched MD'  This implements an
algorithm described in Della Valle and Andersen, J. Chem. Phys.
97, 2682 (1992).  It performs a molecular dynamics simulation
using the 'velocity Verlet' algorithm with the following
modifications.  At each time step and for each particle, the
velocity is reset as follows.  For (f dot v > 0), the velocity is
replaced by the projection of the velocity along the direction of
the force.  For (f dot v <0 ), the velocity is set to zero.  This
will quench the dynamics to the minimum.  This approach seems to
be best suited for getting close to the minimum when the initial
guess may be poor.  Note:  the method of obtaining the atomic mass
is not finalized (waiting for 'real' i/o routines).  It should
treat everything as 1 amu.
\end{list}

\item[relax\_force\_tol:]  This is the stopping criteria for all methods.  The code stops
when the root mean square value of the force components is less
than this value.  default: 1.d-3.

\item[relax\_max\_steps:]  This sets a maximum number of force calls that can be made to
attempt to find the minimum.  Note that in some cases, the actual
number of calls may exceed this slightly since it will always try
to finish the line minimizations in the conjugate gradient
approach.  default: 100

\item[relax\_prefactor:] This is the constant used in relax\_method=1 (see above).  Note
that the efficiency and stability of this method depends on this
choice.  If the value is too large, the optimization will become
unstable.  If it is too small, a large number of force calls is
required to get to the minimum.  default: 1.0

\item[relax\_time\_step:] This is used in relax\_method=3.  It sets the fixed time step for
the MD simulation.  default: 1.0 (probably too small for most
cases.)
\end{list}

\section{Molecular Dynamics}

The code can currently perform either a NVE (constant number,
volume and energy) micro-canonical simulation or a NVT (constant
number, volume and temperature) simulation.  For the later case, a
variety of standard thermostat methods are implemented.  The
initial velocities can be either read from the file
"initial\_velocity" or are set randomly based on an input
temperature.  The integration is performed via the 'velocity
Verlet' algorithm (see any book on MD simulations) with a fixed
time step.  Intermediate atomic positions are output to the file
'md\_output' at every time step.  The masses are currently all set
to 1 amu.  (again waiting for I/O stuff to fix this.)

This is called by the statement "if (run\_moldyn(cfg)) call
diary(cfg)" in socorro.f90.  run\_moldyn returns a logical that
indicates whether any modifications to cfg have occurred.  If an
MD simulation is performed, the cfg returned is that for the final
time step.

The relevant parameters are

\begin{list}{}{}

\item[md\_method:] Specify the molecular dynamics method to be
used.

\begin{list}{}{}
\item[NONE:] (default) - do no MD.
\item[NVE:] perform a NVE simulation using a fixed time-step velocity
Verlet algorithm.
\item[NVT\_RESCALE:] perform a NVT (isochoric,
isothermal) simulation using a fixed time-step.  The temperature
control is through periodic rescaling of the velocities to achieve
the desired temperature.
\item[NVT\_ANDERSON:] perform a NVT
(isochoric, isothermal) simulation using a fixed time-step.  The
temperature is controlled via a stochastic method due to Anderson
(see H C Anderson, J. Chem. Phys. 72, 2384 (1980)).  At each time
step and for each atom, a random velocity from a Maxwell-Boltzman
distribution replaces the velocity with a probability give by
1/temp\_freq.
\item[NVT\_HOOVER:] perform a NVT (isochoric,
isothermal) simulation using a fixed time-step.  The temperature
is controlled using the Hoover implementation of the Nos�
thermostat. (See, for example, "Understanding Molecular
Simulations" by Frenkel and Smit).  The rate of energy flow
between the ions and the heat bath is controlled by
md\_hoover\_mass.
\end{list}

\item[md\_time\_step] Time step used for the MD. See note below regarding units. default: 100.

\item[md\_steps] Number of time steps to integrate the equation of motion. default:
0

\item[md\_skip\_steps]  Number of time steps to ignore before starting to compute averages
- in other words (skip\_steps)*(time\_step) is an equilibration time.  default: 0

\item[md\_init\_temp]  Temperature used to define the initial velocity distributions.
The velocities are selected from a Maxwell-Boltzman distribution.
Then they are adjusted to give zero total momentum and rescaled to give the exact temperature requested. default: 0

\item[md\_desired\_temp] Target temperature for the isothermal simulation methods. default: md\_init\_temp

\item[md\_temp\_freq] Parameter that determines the frequency of velocity modifications for the
isothermal simulation methods.
For NVT\_RESCALE, the velocity is rescaled every md\_temp\_freq time steps.
For NVT\_ANDERSON, it give the inverse of the probability that an atom will get a random velocity in a given time step.

\item[md\_hoover\_mass] Used by NVT\_HOOVER.
It is the effective mass associated with the additional coordinate added to control the temperature.
It may need to be adjusted by trial and error.  default: 1000.

\item[md\_gen\_velocities]  Specify the method for initializing
velocities
\begin{list}{}{}
\item[YES] Determine the initial velocities based on md\_init\_temp.
(default)
\item[NO:] Read the initial velocities from the file
'initial\_velocity' in the run directory.  This file contains a
line with the x, y, and z velocity for each atom on a separate
line.  The order of the atoms is assumed to be the same as in the
crystal file.
\end{list}

\end{list}

\section{Transition State Finding}

Currently, there is only one transition state finding method implemented, the 'dimer method'.  This is described in detail in Henkelman and J�nsson, J. Chem. Phys. 111, 7010 (1999).  This method attempts to find saddle points with one unstable mode (ie a single negative value of the Hessian matrix.)  The trick is to avoid computing the second derivatives of the energy.  To do this, two configurations that differ by a fixed distance are considered.  (This is the 'dimer'.)  The dimer is alternately rotated such that the separation is along the direction of minimum curvature and then translated in the direction of the saddle point.  The current implementation does not incorporate all of the sophisticated algorithms for optimization discussed in the paper. Also, there is currently no way to initialize the first orientation of the dimer other than choosing a direction at random.  (Waiting on the new I/O routines before implementing this.) This takes a large number of forces calls, especially to get the first estimate of the dimer orientation.

This is called by the statement "if (transition\_state(cfg)) call
diary(cfg)" in socorro.f90.  transition\_state returns a logical
that indicates whether any modifications to cfg have occurred.  On
return after a transition state search, cfg corresponds to the
transition state.

The relevant parameters are the following

\begin{list}{}{}

\item[ts\_method]  Specify the transition state method

\begin{list}{}{}
\item[0] (default) do nothing
\item[1] implement the dimer method
\end{list}

\item[dimer\_separation] The magnitude of the real space separation between the two configurations of the dimer.  The algorithms are based on the assumption that this is small.  Of course, if it is too small, numerical errors can become unacceptable because many of the calculations are based on differences between the calculations for each of the two configurations.  default: 0.10

\item[dimer\_force\_tol] Stopping criteria which is based on the rms forces on the 'dimer'.  Ideally, this will correspond to the net forces at the saddle point.  default: 1.d-3

\item[dimer\_max\_steps] Maximum number of calculations of the dimer properties before the algorithm stops.  Note that each calculation of a dimer property requires 2 electronic structure calculations.  In some cases, it may make somewhat more calculations in order to stop at a sensible place.  default: 100
\end{list}

\section{Notes on units:}

The convention in the code is that energies are in Rydbergs and that distances are in Bohrs.
I have made the choice to have the code work with the nuclear masses in units of the electron mass -
i.e. keep in the spirit of atomic units.
For convenience, when masses are entered, they are assumed to be in amu (atomic mass units)
and are converted in the code to electron masses.
Having made this choice for the mass unit, the choice of the time unit is now fixed.  The unit of time is .
Since a typical MD time step is on the order of a few femtoseconds (fs), the typical time steps
will be on the order of 100 in the units used here.

There is a subtlety in the determination of the temperature.  In
classical thermodynamics (MD is classical in the treatment of the
ionic motion), each degree of freedom has a kinetic energy of .
The issue is related to the number of degrees of freedom.  If the
MD method used conserves the total momentum, then the number of
degrees of freedom is 3(N-1).  If the total momentum is not
conserved, the number of degrees of freedom is 3N.  The NVE and
NVT\_RESCALE methods conserve the total momentum.  (Actually, for
NVT\_RESCALE the total momentum remains zero if it starts out
zero. The initial velocities are generated in the code to have
zero total momentum.)  For these methods, the code uses 3(N-1)
degrees of freedom to determine the temperature.  For
NVT\_ANDERSON, the total momentum is not conserved, so the code
uses 3N degrees of freedom to compute the temperature.

\end{document}
